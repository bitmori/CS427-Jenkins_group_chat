%	@file FinalDocumentation.tex
%
%	@author madigan6, ciurej2
%	@date Winter 2013
%	
%   LaTeX source file for the design document for the CS427 project.  More 
%   information about the requirements for this documentation file can be found
%   here: "https://wiki.engr.illinois.edu/display/cs427fa13/Final+Documentation"

\documentclass{article}
\usepackage[parfill]{parskip}		% Package that improves paragraph formatting
\usepackage[pdftex]{graphicx}		% Package that faciliates figure insertion
\usepackage{float}					% Package that allows for arbitrary figure insertion
\usepackage{hyperref}				% Package that allows for text hyperlinking
\usepackage{tikz}					% Package that facilitates box rendering

% The following code block changes a few parameters of the page to make the
% text fill up more space on any given page.  For our liking, the default 'article'
% template is too sparse.
\addtolength{\voffset}{-2cm}
\addtolength{\textheight}{4cm}
\addtolength{\oddsidemargin}{-1.5cm}
\addtolength{\textwidth}{2cm}

% This command creates blue boxes around class names.
% @see "http://tex.stackexchange.com/questions/36401/drawing-boxes-around-words"
%\newcommand\classname[2][]{\tikz[overlay]\node[fill=blue!20,inner sep=2pt, anchor=text, rectangle, rounded corners=1mm,#1] {#2};\phantom{#2}}
%\newcommand\methodname[2][]{\tikz[overlay]\node[fill=green!20,inner sep=2pt, anchor=text, rectangle, rounded corners=1mm,#1] {#2};\phantom{#2}}
\newcommand{\classname}[1] {\texttt{#1}}
\newcommand{\methodname}[1] {\texttt{#1}}
\newcommand{\insertdiagram}[2]
{
	\begin{figure}[H]
		\centering
		\fbox{\includegraphics[height=#2]{figures/#1}}
		\caption{UML Diagram for the #1 Class}
	\end{figure}
}

\begin{document}

	% Title Page %
	\title{Chat Plugin Design Document}
	\author{Timothy Madigan and Joseph Ciurej \\
		University of Illinois: Urbana-Champaign (CS427)}
	\date{\today}
	\maketitle

	% Table of Contents %
	\tableofcontents
	\clearpage

	% Body %
	\section[Plugin Overview]{A Brief Overview of the Chat Plugin}
	The purpose of this project is to create an extension to the continuous 
	integration server Jenkins that facilitates chat communication between
	users of the server.  The main feature of this extension is an augmentation
	to the standard Jenkins user web pages that allows for messages to be 
	transferred between users connected to one or more of these pages through
	a dynamically updating web interface.

		\subsection[Motiviation]{Motivation for Chat Plugin}
		The primary purpose of this plugin is to enable quick and easy 
		communication between Jenkins users through the provision of an 
		integrated chat client as an augmentation to the Jenkins web interface.
		While other chat clients exist in abundance, the Jenkins chat plugin
		allows for easy integration into a familiar environment with minimal
		installation difficulties.

		\subsection[Project Goals]{Goals of Chat Plugin}
		The main goal of this project is to provide an intuitive, feature-rich
		chat client within the Jenkins web interface.  By integrating a chat client
		into the Jenkins interface, we aim to remove the need for extraneous chat 
		applications, thus providing a more seamless workflow.  Additionally,
		eliminating the need for extra applications minimizes dependencies on
		additional exterior applications, decreasing both maintenance and
		integration costs for users.

		In an attempt to enhance user satisfaction, we built the application with
		a variety of features, including:

		\begin{itemize}
			\item Peer-to-peer communication between users via a familiar chat
			interface.
			\item Group chat support to allow users to communicate with multiple
			team members simultaneously.
			\item Identification system that allows for user-specified names by
			both system administrators (via an admin file) and chat clients
			(via a user interface in the chat client).
			\item Persistence of chat logs to allow for users and management to 
			cross-reference past communications.
		\end{itemize}
		
		Our hope is that these features will serve to facilitate efficient
		communication between users and provide near-complete coverage of desired
		functionality.


	\section[Architecture and Design]{Chat Plugin Architecture and Design}
	The overall architecture of the Jenkins chat plugin follows a traditional
	\href{http://en.wikipedia.org/wiki/Client\%E2\%80\%93server\textunderscore model}{client-server model} 
	where clients to the chat plugin send messages to
	a back-end server and this back-end server forwards incoming messages 
	based on provided recipient information.  The implementation for the server 
	component of the model (and the implementation for supporting back-end
	functionality) is programmed in Java while the client component is
	programmed with a combinaion of Javascript and HTML.  These two components
	communicate using websocket technology (provided by the 
	\href{https://github.com/mrniko/netty-socketio/releases}{Netty-SocketIO} library), 
	which allows for seamless and asynchronous client-server interaction.
	The plugin extends Jenkins to closely integrate the start-up and shut-down
	processes of the server component with the Jenkins server and to provide
	server access to clients through the Jenkins web interface.

		\subsection[Primary Components]{Primary Plugin Library Components}
		The core of the Jenkins chat plugin consists of the types used to
		implement the server and client components of the client-server model.
		For each of these primary components, the listing below describes the
		major functionality encapsulated by that primary component and provides
		a visual aid for its role in the project through a 
		\href{http://www.csci.csusb.edu/dick/samples/uml0.html}{UML diagram}.

			\subsubsection[\classname{ChatServer}]{The \classname{ChatServer} Component}
			The primary type that implements the server 
			component in the chat plugin.  This type encapsulates the behavior of 
			the server component in the client-server model for the chat plugin, 
			transporting received client messages across the network to other 
			designated clients.

			\insertdiagram{ChatServer}{1.1in}

			The primary functions of this type are client connection/disconnection
			broadcast updates (which are performed to inform all connected clients
			when a particular client connects or disconnects), client message
			forwarding (performed whenever a message is receieved from a client
			with a destination address), and back-end to front-end message forwarding.

			\subsubsection[\classname{ClientListing}]{The \classname{ClientListing} Component}
			A supplemental type that implements a
			mapping from addresses of chat clients to the stored names of these
			clients.  This type is used primarily by the \classname{ChatServer} type
			in order to retrieve information about a particular client given the
			remote address of this client, which aids the server in associating
			unique names with arbitrary source/destination clients.  This type
			is also used to persist user name data between chat sessions and allow
			this name information to be specified through a simple format (i.e.
			the \href{http://en.wikipedia.org/wiki/Comma-separated\textunderscore values}{CSV file format}).

			\insertdiagram{ClientListing}{2.0in}

			Essentially, the \classname{ClientListing} type is a two-way hashtable
			that associates network addresses with user names and supports data
			export to and data import from CSV files.

			\subsubsection[\classname{UserGroup}]{The \classname{UserGroup} Component}
			Another supplemental type to the 
			\classname{ChatServer} type that represents a hierarchical group ordering
			structure.  This type is used to group chat clients within 
			uniquely-named sets of communication groups based on user-specified 
			group structuring.  This grouping and ordering information is utilized
			by the \classname{ChatServer} type to determine the recipients of a 
			message given an identifier for a group of users.  Additionally, the
			\classname{GroupDatabase} type provides support to store the group 
			structuring information contained in a \classname{UserGroup} 
			instance into a database for persistent storage (with similar
			functionality available for loading this information).

			\insertdiagram{UserGroup}{1.25in}

			A single instance of the \classname{UserGroup} type represents the
			subroot of a group hierarchy tree structure.  Each node within this
			structure represents a distinct group where the proper descendents of
			any node represent disjoint, strict sub-groups of the group 
			represented by the node itself.

			\subsubsection[\classname{ChatHistory}]{The \classname{ChatHistory} Component}
			A type that facilitates the storage of
			chat messages within a database with a scheme that supports efficient 
			querying based on recipient group and author.  The primary functionality 
			of this type is storing given chat messages into a database and querying 
			these messages based on author or group.
			\insertdiagram{ChatHistory}{0.50in}
			While not fully integrated at the point of initial release, this
			type is intended to be integrated with the \classname{ChatServer}
			type to support the persistence of chat logs between chat clients
			using the Jenkins chat plugin.

			\subsubsection[\classname{ChatClient}]{The \classname{ChatClient} Component}
			The non-explicit, unencapsulated 
			conglomerate of code that comprises the client portion of the 
			client-server model.  This portion of the implementation is responsible
			for displaying connected user information based on data sent from
			the \classname{ChatServer}, properly populating text areas based on
			messages passed from the \classname{ChatServer}, and sending messages
			to the \classname{ChatServer} based on user input.

			This code is injected into every Jenkins web view to facilitate
			client communication to the back-end chat server from any point
			in the Jenkins web interface.  This injection is performed by way
			of a Jenkins \classname{PageDecorator} extension, which is described
			in the next section.

			For additional information about the client component of the client-
			server model, see the \textbf{plugin/src/main/webapp} subdirectory 
			of the project source (relative to the base directory of the plugin).

		\subsection[Jenkins Extensions]{Jenkins Plugin Extension Points}
		The initial release of the Jenkins chat plugin utilizes three Jenkins
		extension points in order to enable the plugin's basic chat functionality.
		The following list describes our plugin's extensions to Jenkins and the
		service that these extensions provide:

		\begin{itemize}
			\item \classname{ChatClientPlugin} extends the \classname{Plugin} extension
			point of Jenkins in order to define start-up and shut-down functionality
			for the chat plugin, which allows for a \classname{ChatServer} instance
			to be created on Jenkins initialization.
			\item \classname{ChatClientDecorator} extends the \classname{PageDecorator} 
			extension point of Jenkins to facilitate the injection of arbitrary
			Javascript and HTML into the Jenkins web interface pages.  Exercising this 
			feature allows the chat client interface to be displayed on all Jenkins 
			web pages.
			\item \classname{ChatClientRunListener} extends the \classname{RunListener}
			extension point to capture build submission and completion information,
			which the type relays to the running \classname{ChatServer} as a message
			to be forwarded to all connected chat clients (though it's important to
			note that this second function isn't implemented for the initial release).
		\end{itemize}

		\subsection[Control Flow]{Example of Chat Plugin Control Flow}
		In order to aid the reader in understanding the flow of the chat plugin
		and how exactly it's integrated into Jenkins, this section contains a
		high-level step-by-step example execution of the plugin.  By stepping
		through this example, the user should be able to get a better idea
		about control flow within the plugin without needing to open up a debugger
		and step through the execution manually.

			\subsubsection[Step-by-Step Execution]{Step-by-Step Plugin Execution}
			The following is a high-level step-by-step execution of the logic
			within the chat plugin starting from Jenkins server initialization
			and ending at Jenkins shut down:

			\begin{itemize}
				\item Upon instantiating an instance of the Jenkins server,
				the Jenkins core scans through all installed plugins for 
				extensions to the \classname{Plugin} type.  As an extension to
				this type, the \classname{ChatClientPlugin} type will be discovered
				and its \methodname{start} function will be invoked.
				\begin{itemize}
					\item Within the \methodname{start} function, an instance of the 
					\classname{ChatServer} type is instantiated, which causes a server
					socket to be opened on a port on the local machine that listens 
					for messages and connection requests from clients.
				\end{itemize}
				\item When a Jenkins user opens up a Jenkins web interface view,
				the \classname{ChatClientDecorator} extension intercepts the view
				before it's serviced and injects the chat client interface into
				the view.
				\item After receiving the Jenkins web view, the injected web code
				performs a connection request to the running \classname{ChatServer}
				instance.
				\begin{itemize}
					\item Upon receiving a client connection request, the 
					\classname{ChatServer} resolves the name of the remote host by
					querying its \classname{ClientListing}, sending the client's 
					associated name as a connection response.  Additionally, the
					\classname{ChatServer} will broadcast a message to all connected 
					clients indicating that the new client has connected.
					\item On getting a connection message, the chat client web
					interface updates the view to indicate that the client
					referenced in the message has connected.
				\end{itemize}
				\item When a client closes its serviced Jenkins web view, the
				injected web code sends a disconnect message to the 
				\classname{ChatServer}.
				\begin{itemize}
					\item Upon receiving a client disconnect message, the 
					\classname{ChatServer} first associates the remote host with a
					name by queryings its \classname{ClientListing}, then immediately
					broadcasts a message to all connected clients that indicates 
					that the given client has disconnected.
					\item On getting a disconnect message, the chat client web
					interface updates the view to indicate that the client
					referenced in the message has disconnected.
				\end{itemize}
				\item When a client enters a message into a chat box, the chat
				client interface code sends the entered text and destination
				client as a packet to the \classname{ChatServer}.
				\begin{itemize}
					\item When the \classname{ChatServer} receives a client
					message packet, it determines the host name of the destination
					client and relays the given packet to this client.
					\item Upon receiving a message from the \classname{ChatServer},
					the client interface populates the corresponding chat box
					with the contents of the given message.
				\end{itemize}
				\item At the time the Jenkins server instance is shut down, it again
				searches through all installed plugins for \classname{Plugin}
				extensions, calling the \methodname{stop} method for each such
				extension.
				\begin{itemize}
					\item During its own \methodname{stop} method invocation, the
					\classname{ChatClientPlugin} type calls the \methodname{shutdown}
					method for the \classname{ChatServer}.  This action causes a
					broadcast to be sent out to all connected clients indicating
					that the server has been terminated, followed by the closure
					of the server socket on the local machine.
				\end{itemize}
			\end{itemize}

			\subsubsection[Notes about Execution]{Important Notes about the Execution Process}
			Only the most important features of the chat plugin have their
			corresponding executions detailed in the step-by-step explanation
			in the previous section.  Other more minor features (such as
			chat client build notifications and user interface functionality)
			haven't been detailed, but their role in the control flow of the
			chat plugin can be gleaned by skimming the documentation for
			the files associated with these features.  
			
			Also, important features added after the initial release are not 
			included in the execution diagram in this document, but the control 
			flow explanation in future documentation will be amended to account 
			for this fact.


	\section[Future Plans]{Plans for the Future of the Chat Plugin}
	During our initial development period, we found that we were not able to
	fully realize many of the features we wished to incorporate into the
	Jenkins chat plugin.  The following section details the major additions
	we wish to integrate into the chat plugin in the future as well as future
	plans we have for the chat plugin project.

		\subsection[Additional Features]{Listing of Additional Chat Plugin Features}
		There were many features that the team considered during the initial
		development period that couldn't be fully implemented due to time
		constraints.  Many of these features enhance user experience by providing
		a more streamlined interface, additional communication features, and
		superior data persistence.

		The following is a listing of all the additional chat plugin features
		that we hope to implement in the future:

		\begin{itemize}
			\item Complete integration of group chat functionality into the
			front-end web interface.
			\item Automatic group recognition for chat clients and automated 
			joining of group chats on page load for the connecting user.
			\item Adding functionality to the front-end interface to allow for
			user name and contained group specification for unrecognized clients.
			\item Transferring user name to client IP address mappings to a 
			database storage scheme on the Jenkins-hosting machine from current 
			local file storage scheme.
			\item Full integration of the database storage functionality within
			the plugin to support efficient persistence of chat, user, and
			group information.
			\item Notification system that sends build messages to relevant clients
			whenever a Jenkins build is submitted or completed. 
			\item Extension of the build notification functionality to allow for
			more granular specification of build information recipients.
			\item Adding the ability for users to set their status so that other
			connected clients have the ability to gauge their availability at
			a glance.
			\item Automatic updating of user status based on client activity
			within the Jenkins web interface.
			\item Redesign of the front-end interface to more closely resemble
			that of Facebook with a perpetual bottom action bar that contains
			open chats and connected clients.
		\end{itemize}

		\subsection[Project Publication]{Goals for Releasing the Chat Plugin}
		Down the road, we hope to publicly release the source for the chat
		plugin by uploading the project to GitHub.  This way, any team member
		or curious open-source contributor may build upon the project in any
		way they wish.  Additionally, we asprie to eventually submit the
		chat plugin as an official Jenkins extension, most likely a few weeks
		after the project becomes open-source.


	\section[Appendix]{Miscellanous Chat Plugin Information}
	The following section contains extra information about the chat plugin,
	such as installation steps and steps to execute various other plugin
	utilities.  This section should be referenced when searching for extra
	instructions for the installation and maintenance of the chat plugin.

		\subsection[Plugin Installation]{Steps for Chat Plugin Installation}
		Installation for the chat plugin has the following dependencies, which
		must be installed before the plugin itself can be installed:

		\begin{description}
			\item[\href{http://www.oracle.com/technetwork/java/javase/downloads/jdk7-downloads-1880260.html}{Java Development Kit}] (v1.7+) \hfill \\
			A version of the standard Java development kit at least as recent 
			as version 1.7 is needed primarily to import extended networking 
			functionality only in later JDK versions.

			\item[\href{http://jenkins-ci.org/}{Jenkins}] (v1.509.3+) \hfill \\
			A version of Jenkins more recent than version 1.509.3 is needed
			to support the chat server plugin (older versions may work, but they
			have not been tested).

			\item[\href{https://github.com/mrniko/netty-socketio/releases}{Netty-SocketIO}] (v1.5.2+) \hfill \\ 
			An open source Java library that facilitates network communication 
			via websockets in Java.  This library is used by the chat plugin
			to aid in the transferral of mesages to and from chat clients.
		\end{description}

		After these dependencies are installed, installing the chat plugin itself
		is relatively straightforward.  The steps for performing this installation
		are as follows:

		\begin{enumerate}
			\item Navigate to the \textbf{plugin} subdirectory from the chat
			plugin source base directory.

			\item Run the script within this directory by the name of 
			\textbf{export.sh}.
		\end{enumerate}

		After performing the second step, a sequence of installations should
		follow.  After all these subsequent installations are complete, the
		plugin should be fully installed and accessible through Jenkins.

		\subsection[Plugin Execution]{Steps for Chat Plugin Execution}
		By the nature of the plugin, the majority of the execution occurs when
		booting up a Jenkins server instance and opening up one of the Jenkins
		user web pages.  That being said, there are no explicit steps for
		executing the plugin apart from the installation, which is detailed
		in the previous section.

		\subsection[Plugin Test Execution]{Steps for Running Implementation Tests}
		Running the automated tests for the chat plugin can be performed by
		following these steps:

		\begin{enumerate}
			\item Navigate to the \textbf{lib} subdirectory from the chat
			plugin source base directory.

			\item Perform a Maven installation within this directory.  This
			can be performed on Unix machines by running the command
			\texttt{`mvn install'}.
		\end{enumerate}

		The manual tests for the chat plugin can be found within the 
		\textbf{lib/test} subdirectory of the base plugin directory.  These
		tests can be executed by simply following the instructions contained
		within each test document.

		\subsection[Plugin Document Generation]{Steps for Generating Implementation Documentation}
		Generating the Javadoc documentation files for the chat plugin can
		be performed by following these steps:

		\begin{enumerate}
			\item Navigate to the \textbf{lib} subdirectory from the chat
			plugin source base directory.

			\item Perform a Maven Javadoc generation within this directory.  
			This can be performed on Unix machines by running the command
			\texttt{`mvn javadoc:javadoc'}.
		\end{enumerate}

		The Javadoc documentation files will be placed within the subdirectory
		\textbf{target/site/apidocs} relative to the \textbf{lib} directory.  
		These files are most easily viewed by opening the file 
		\textbf{target/site/apidocs/index.html} within a web browser.
\end{document}
